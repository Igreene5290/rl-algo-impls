\documentclass{article}
\usepackage{iclr2024_conference}
\begin{document}
\title{A Competition Winning Deep Reinforcement Learning Agent in microRTS}
\author{Scott Goodfriend$^{1}$ \\
$^{1}$ goodfriend.scott@gmail.com \\
}
\maketitle
\begin{abstract}
    Scripted solutions with pathfinding algorithms have predominantly won the five
    previous iterations of the IEEE microRTS ($\mu$RTS) competitions hosted at CIG and
    CoG. Despite Deep Reinforcement Learning (DRL) algorithms making significant strides
    in RTS-style games, their adoption has been limited in this primarily academic
    competition due to the considerable training resources required and the complexity
    inherent in creating and debugging such agents. The RAISocketAI, as part of the
    IEEE-CoG2023 microRTS competition, has emerged as the first winning DRL agent. In a
    non-performance constrained environment, RAISocketAI regularly defeated the two
    preceding competition winners. This first competition-winning DRL submission can be
    a benchmark for future microRTS competitions and a starting point for future DRL
    research. In particular, this submission successfully used fine-tuning a base model
    to perform significantly better on specific maps. Further work in Behavior Cloning
    has proven promising as an even more economical way to bootstrap a model.
\end{abstract}
\section{Introduction}
Deep reinforcement learning has proven to be a powerful tool for solving complex
problems requiring several steps to achieve a goal, such as Atari games \citep{DBLP:journals/corr/MnihKSGAWR13}, continuous
control tasks \citep{DBLP:journals/corr/LillicrapHPHETS15}, and even real-time strategy
(RTS) games. AlphaStar is a famous example of an agent trained with DRL on StarCraft
II to defeat professional players \citep{Vinyals2019GrandmasterLI}. However, AlphaStar was trained with thousands of
CPUs and GPUs/TPUs for several weeks. RTS games are particularly challenging for DRL for
several reasons:
\begin{enumerate}
    \item the state and action spaces are large and varied with different terrain and
        unit types;
    \item each unit type can have different actions and abilities;
    \item each action can control several units;
    \item rewards are sparse (win, loss, or tie) and delayed by possibly several
    thousand timesteps;
    \item winning requires combining tactical (micro) and strategic (macro) decisions;
    \item actions must be taken in real-time (i.e., the game won't wait for the agent to
        take an action);
    \item the agent might not have full visibility of the game state (i.e., fog of war); and
    \item events in the game might be non-deterministic.
\end{enumerate}

microRTS (stylized as $\mu$RTS) is a minimalist, open-source RTS game testbed designed for research
purposes \citep{Ontan2013TheCM}. It includes many aspects of RTS games, simplified: different unit types, unit-specific
actions terrain, resource collection and consuptiom to build units, and unit-to-unit combat
where units have different strengths and weaknesses. microRTS also supports fog of war
and non-determinism; however, these were disabled for the IEEE-CoG2023 microRTS
competition.

The IEEE microRTS competition has been hosted at the Conference on Games (CoG) nearly
every year since 2019 and at the Conference on Computational Intelligence and Games
(CIG) before that since 2017 \citep{Ontañón_Barriga_Silva_Moraes_Lelis_2018}.
Competitors submit an agent that plays against other agents in a round-robin tournament
on 12 maps with different distributions of terrain, resources, and starting units and
buildings. 8 "Open" maps are known beforehand and are used to determine the competition winner.
The other 4 "Hidden" maps are not revealed to participants until after the competition.
These hidden maps are meant to test the generalization ability of the agents, but aren't
part of the competition to allow organizers to participate. Hidden map results are also
released.

This paper describes the RAISocketAI agent, which won the IEEE-CoG2023 microRTS
competition and is the first DRL agent to win a microRTS competition. RAISocketAI
extends the work of MicroRTS-Py (formally Gym-$\mu$RTS/Gym-MicroRTS) to be competitive
in the Open competition maps. MicroRTS-Py is an OpenAI Gym environment for microRTS,
which makes it easier to train DRL agents \citep{DBLP:journals/corr/abs-2105-13807}.
MicroRTS-Py also includes a DRL training framework using Proximal Policy Optimization
(PPO) \citep{DBLP:journals/corr/SchulmanWDRK17}, state and action space vectorization,
invalid action masking, and environment vectorization.
\citet{DBLP:journals/corr/abs-2105-13807} only trained an agent for one of the smaller
maps. RAISocketAI extends the DRL training framework to be competitive on other, more
complicated maps.

Models in the IEEE microRTS competition are supposed to submit actions for each timestep
within 100ms. Without GPU acceleration, this is a significant constraint for deep neural
network agents. Being a Python agent, RAISocketAI also had to communicate with the Java
microRTS efficiently, which required optimizing the data representation.

Despite these constraints, RAISocketAI won the IEEE-CoG2023 microRTS competition. The
agent consists of 7 trained neural networks. It took about 70 GPU-days to train these
models, which is within budget for researchers, though possibly outside of budget for
students. However, 4 of these neural networks were trained from an existing model. These
transfer learning models were used to make the models perform better on specific maps.
These transfer learning models took in total 18 GPU-days to train.

While microRTS doesn't support human players. The IEEE microRTS competition means there
are several agents available to use imitation learning on. Work following the
competition shows that behavior cloning and fine-tuning with PPO can be used to train a
competitive agent more economically. Using the same playthroughs to train the critic
heads on win-loss rewards means that PPO can be trained with just sparse win-loss
rewards, eliminating the need for a difficult to tune human reward function.

\section{Related Work}


\section{Method}
\subsection{RAISocketAI}

\bibliography{iclr2024}
\bibliographystyle{iclr2024_conference}
\end{document}